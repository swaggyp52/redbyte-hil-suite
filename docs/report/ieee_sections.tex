% ==========================================
% IEEE Senior Design Report Sections
% generated by Antigravity
% ==========================================

\section{Abstract}
Grid-forming inverters operating as Virtual Synchronous Machines (VSMs) require rigorous validation to ensure stability in low-inertia microgrids. This project presents the \textit{GFM HIL Suite}, a Python-based telemetry and simulation platform designed for Hardware-In-the-Loop (HIL) testing of VSM algorithms. The system provides real-time oscilloscope-style visualization, deterministic data recording, and automated session comparison. By leveraging a multi-threaded architecture and a custom JSON telemetry protocol, the specific software achieves sub-20ms latency and reliable logging of high-frequency power data. Experimental results verify the platform's ability to capture transient fault events and quantify controller performance using Root Mean Square Error (RMSE) metrics.

\section{Introduction}
As renewable energy penetration increases, the grid loses physical inertia provided by synchronous generators. VSM control algorithms address this by emulating inertia in software. Validating these complex controls on high-voltage hardware poses significant risks; thus, HIL simulation is the standard testing method. \par
We developed a dedicated desktop telemetry suite to interface with VSM microcontrollers. The objectives were to:
\begin{enumerate}
    \item Provide real-time visualization of 3-phase voltage and current.
    \item Enable deterministic recording and playback of test scenarios.
    \item Automate the verification of control responses against baselines.
\end{enumerate}

\section{System Architecture}
The software follows a modular desktop-application architecture inspired by modern OS metaphors (RedByte OS), implemented in Python using PyQt6.

\subsection{Data Ingestion Layer}
A threaded \texttt{SerialReader} manages UART communication with the HIL hardware. It enforces a line-based JSON protocol for telemetry frames, ensuring thread safety via Qt Signals/Slots to decouple data ingestion from the GUI event loop. This design prevents UI freezing even under high data loads (100Hz+).

\subsection{Data Management}
Telemetry is stored in ``Data Capsules''---JSON structures containing both metadata (firmware version, scenario ID) and time-series arrays. The \texttt{Recorder} module buffers incoming frames and flushes to disk upon session completion, ensuring zero data loss. The \texttt{Replayer} engine allows historical sessions to be injected back into the visualization pipeline with preserved timing.

\subsection{User Interface}
The graphical interface uses a Multi-Document Interface (MDI) to host concurrent tools:
\begin{itemize}
    \item \textbf{Monitor App}: A \texttt{pyqtgraph}-based oscilloscope for real-time signal plotting.
    \item \textbf{Scenario Runner}: A tool to inject event markers (e.g., ``Voltage Sag'') for test coordination.
    \item \textbf{Analysis Engine}: A post-processing tool that computes statistical deviations (RMSE, Max Delta) between reference and test runs.
\end{itemize}

\section{Testing and Results}
The system was validated using a comprehensive test suite covering unit logic and end-to-end integration.

\subsection{Performance Metrics}
Latency tests confirmed the system processes and renders telemetry frames within 10ms on standard hardware. The circular buffer implementation enables continuous monitoring of long-duration tests (1 hour+) without memory leaks.

\subsection{Validation Capabilities}
To verify VSM inertia settings, we simulated a ``Voltage Sag'' scenario. The platform recorded the baseline response and a subsequent test run. The built-in Analysis Engine computed an RMSE of 0.0 for identical runs and correctly identified induced deviations in faulty runs.
\begin{equation}
RMSE = \sqrt{\frac{1}{N}\sum_{i=1}^{N}(\hat{y}_i - y_i)^2}
\end{equation}
This automated comparison drastically reduces the time required for regression testing of new firmware builds.

\section{Conclusion}
The \textit{GFM HIL Suite} successfully meets all MVP requirements, providing a robust, student-maintainable tool for VSM research. Future work includes integrating OPAL-RT API automation for fully closed-loop scenario testing.
